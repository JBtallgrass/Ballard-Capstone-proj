\documentclass[12pt]{article}
\usepackage[margin=1in]{geometry}
\usepackage{hyperref}
\usepackage{enumitem}
\usepackage{graphicx}
\usepackage{amsmath}
\usepackage{setspace}
\usepackage{titlesec}
\usepackage{fancyhdr}
\usepackage{biblatex}
\addbibresource{references.bib}

\title{Bridging the Gap: A Data-Driven Analysis of Education and Employment Alignment in North Central Arkansas}
\author{}
\date{}

\begin{document}

\maketitle

\begin{abstract}
(Wait until the project is complete.)
\end{abstract}

\section{Introduction}

\subsection{Domain and Topic Selection}
 

\subsection{The Data Problem}

The data problem is thus twofold:
\begin{enumerate}[label=\arabic*.]
  \item Can we quantify the alignment between educational program completions (CIP codes) and local job demand (SOC codes)?
  \item Can we identify geographic and structural disparities that influence education-to-employment transitions across rural counties?
\end{enumerate}

\subsection{Data Sources}
The project leverages a diverse range of public, open-source datasets, including:
\begin{itemize}
  \item U.S. Census Bureau \href{https://data.census.gov}{(data.census.gov)}
  \item Bureau of Labor Statistics \href{https://www.bls.gov/data/}{(bls.gov)}
  \item IPEDS \href{https://nces.ed.gov/ipeds/}{(nces.ed.gov)}
  \item Arkansas Workforce Services \href{https://www.discover.arkansas.gov}{(discover.arkansas.gov)}
  \item O*NET and CIP-SOC Crosswalk \href{https://www.onetonline.org}{(onetonline.org)}
  \item FCC Broadband \href{https://broadbandmap.fcc.gov}{(fcc.gov)}
\end{itemize}

\subsection{Project Implementation Steps}
This project follows a modified CRISP-DM methodology:
\begin{enumerate}[label=\arabic*.]
  \item Business Understanding
  \item Data Collection
  \item Data Preparation
  \item Exploratory Data Analysis (EDA)
  \item Modeling
  \item Evaluation
  \item Deployment
\end{enumerate}

\subsection{Key Components and Expected Results}
Key components include:
\begin{itemize}
  \item County-level mismatch analysis
  \item Skills pipeline visualization tool
  \item Predictive modeling
  \item Disparity mapping
  \item Policy recommendations
\end{itemize}

\subsection{Project Limitations}
This project excludes:
\begin{itemize}
  \item Individual-level tracking
  \item Real-time job posting data
  \item Causal inference
  \item Private training data
\end{itemize}

\section{Regional and Theoretical Background}

\subsection{Overview of North Central Arkansas}
\textbf{Problem Context.}  

\textbf{Demographic and Economic Structure.} 
\subsection{Education and Workforce Policy Context}
\textbf{State-Level Initiatives and Funding.} 

\textbf{Role of Community Colleges and Workforce Boards.} 

\subsection{Conceptual Framework}
\begin{itemize}
  \item \textbf{Human Capital Theory}
  \item \textbf{Labor Market Signaling}
  \item \textbf{Rural Development and Digital Equity}
\end{itemize}

\section{Methodology: CRISP-DM Framework}
\subsection{Business Understanding}

\subsection{Data Understanding}

\subsection{Data Preparation}

\subsection{Modeling}

\subsection{Evaluation}

\subsection{Deployment}

\section{Data Sources and Processing}
Summary of dataset types, use of crosswalks, handling missing values

\section{Exploratory Data Analysis (EDA)}
\subsection{Education Pipeline Trends}

\subsection{Labor Market Insights}

\subsection{Equity and Infrastructure Factors}


\section{Predictive Modeling and Results}
\subsection{Enrollment and Workforce Forecasting}

\subsection{Mismatch Analysis}
O
\subsection{Regional Disparities}


\section{Interpretation and Discussion}


\section{Recommendations}
\subsection{Institutional}

\subsection{Policy-Level}

\subsection{Community Engagement}


\section{Conclusion}


\section{Appendices}
\begin{itemize}
  \item A. Data dictionary
  \item B. Code/model summaries
  \item C. County-level snapshots
  \item D. Methodological notes
\end{itemize}


\end{document}
